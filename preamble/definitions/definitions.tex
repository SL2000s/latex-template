% -----------------------------------------------------------------------------
% File: definitions.tex
% Description: This file contains custom LaTeX definitions, macros, and
%              environments used throughout the document. It is part of the
%              preamble setup for the LaTeX template.
% -----------------------------------------------------------------------------

% Ensure this file can only be included once
\ifdefined\definitionsLoaded
  \endinput
\else
  \def\definitionsLoaded{}
\fi

% ------------ Custom Macros ------------
\newcommand{\R}{\mathbb{R}} % Short for real numbers
\newcommand{\N}{\mathbb{N}} % Short for natural numbers
\newcommand{\set}[1]{\{#1\}} % Set notation
\newcommand{\inner}[2]{\langle #1, #2 \rangle} % Inner product
\newcommand{\todo}[1]{\textcolor{red}{\textbf{TODO:} #1}} % Easy-to-spot TODO notes

\DeclareMathOperator*{\argmax}{arg\,max} % Argmax
\DeclareMathOperator*{\argmin}{arg\,min} % Argmin

% \DeclareSIUnit\ar{år} % Needs \usepackage{siunitx}, use with \SI{50}{\ar}

\newcommand\norm[1]{\left\lVert#1\right\rVert} % Norm with double vertical bars
\newcommand\normx[1]{\left\Vert#1\right\Vert} % Alternative norm with single vertical bars

% ------------ Custom Environments ------------
\theoremstyle{definition}
\newtheorem{definition}{Definition}[section]
\newtheorem{example}{Example}[section]

\theoremstyle{plain}
\newtheorem{theorem}{Theorem}[section]
\newtheorem{lemma}[theorem]{Lemma}
\newtheorem{corollary}[theorem]{Corollary}

% ------------ Renew Commands ------------
%% Align comments and use "//" in algorithms (source: https://tex.stackexchange.com/questions/180212/how-to-align-comments-in-algorithm-code)
% \renewcommand{\Comment}[2][.5\linewidth]{\leavevmode\hfill\makebox[#1][l]{//~#2}}
