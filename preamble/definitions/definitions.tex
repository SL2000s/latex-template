% -----------------------------------------------------------------------------
% File: definitions.tex
% Description: This file contains custom LaTeX definitions, macros, and
%              environments used throughout the document. It is part of the
%              preamble setup for the LaTeX template.
% -----------------------------------------------------------------------------

% Ensure this file can only be included once
\ifdefined\definitionsLoaded
  \endinput
\else
  \def\definitionsLoaded{}
\fi

% ------------ Custom Macros ------------
\newcommand{\R}{\mathbb{R}} % Short for real numbers
\newcommand{\N}{\mathbb{N}} % Short for natural numbers
\newcommand{\Z}{\mathbb{Z}} % Short for integers
\newcommand{\set}[1]{\{#1\}} % Set notation
\newcommand{\inner}[2]{\langle #1, #2 \rangle} % Inner product
\newcommand{\todo}[1]{\textcolor{red}{\textbf{TODO:} #1}} % Easy-to-spot TODO notes

\DeclareMathOperator*{\argmax}{arg\,max} % Argmax
\DeclareMathOperator*{\argmin}{arg\,min} % Argmin

\newcommand\mintwo[2]{\min\!\left(#1\!,\;#2\right)} % min with parentheses
\newcommand\Mintwo[2]{\(\min\!\left(#1\!,\;#2\right)\)} % min with parentheses


% \DeclareSIUnit\ar{år} % Needs \usepackage{siunitx}, use with \SI{50}{\ar}

\newcommand\norm[1]{\left\lVert#1\right\rVert} % Norm with double vertical bars
\newcommand\abs[1]{\left\lvert#1\right\rvert} % Norm with single vertical bars

\newcommand{\bigO}[1]{\mathcal{O}\!\left(#1\right)} % Inline Big O notation
\newcommand{\BigO}[1]{\(\mathcal{O}\!\left(#1\right)\)} % Big O notation
\newcommand{\bigTheta}[1]{\Theta\!\left(#1\right)} % Inline Big Theta notation
\newcommand{\BigTheta}[1]{\(\Theta\!\left(#1\right)\)} % Big Theta notation
\newcommand{\bigOmega}[1]{\Omega\!\left(#1\right)} % Inline Big Omega notation
\newcommand{\BigOmega}[1]{\(\Omega\!\left(#1\right)\)} % Big Omega notation
\newcommand{\poly}[1]{\operatorname{poly}\!\left(#1\right)} % Inline polynomial complexity
\newcommand{\Poly}[1]{\(\operatorname{poly}\!\left(#1\right)\)} % polynomial complexity
\newcommand{\polylog}[1]{\operatorname{polylog}\!\left(#1\right)} % Inline polylogarithmic complexity
\newcommand{\Polylog}[1]{\(\operatorname{polylog}\!\left(#1\right)\)} % polylogarithmic complexity

\newcommand{\cmark}{\ding{51}}  % Check mark from pifont
\newcommand{\xmark}{\ding{55}}  % Cross mark from pifont

\newcommand{\llpar}{\left(}
\newcommand{\rrpar}{\right)}
\newcommand{\parantheses}[1]{\llpar#1\rrpar}
\newcommand{\llbrack}{\left[}
\newcommand{\rrbrack}{\right]}
\newcommand{\brackets}[1]{\lbrack#1\rbrack}
\newcommand{\llbrace}{\left\lbrace}
\newcommand{\rrbrace}{\right\rbrace}
\newcommand{\braces}[1]{\lbrace#1\rbrace}
\newcommand{\llangle}{\left\langle}
\newcommand{\rrangle}{\right\rangle}
\newcommand{\angles}[1]{\langle#1\rangle}

% ------------ Custom Environments ------------
\theoremstyle{definition}
\newtheorem{definition}{Definition}[section]
\newtheorem{example}{Example}[section]

\theoremstyle{plain}
\newtheorem{theorem}{Theorem}[section]
\newtheorem{lemma}[theorem]{Lemma}
\newtheorem{corollary}[theorem]{Corollary}

% ------------ Renew Commands ------------
%% Align comments and use "//" in algorithms (source: https://tex.stackexchange.com/questions/180212/how-to-align-comments-in-algorithm-code)
% \renewcommand{\Comment}[2][.5\linewidth]{\leavevmode\hfill\makebox[#1][l]{//~#2}}
